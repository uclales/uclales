\author{Thijs Heus}
\lecture[Statistics]{Statistics and output}{statistics}
\begin{frame}[allowframebreaks]{Output files}
 \begin{itemize}
  \item Restart files \code{*.rst} only for internal model use. Output every \code{frqhis} seconds
  \item 3D output files \code{name.nc}: 3D output of the main quantities. Output done every \code{frqanl} seconds. Bulky!
  \item 2D Crosssections \code{name.out.cross*nc}: Crosssections of the data in th xy, xz, yz planes, as well as 2D integrated quantities like Liquid Water Path. Output done every \code{frqcross} seconds, governed by \code{lcross}, \code{lxy}, \code{lxz}, \code{lyz}
  \item 1D Profiles \code{name.ps*nc}. Profiles as a function of height. Output every \code{savg\_intvl}, sampling every \code{ssam\_intvl}. Need to be post processed for MPI runs.
  \item Timeseries \code{name.ts.*nc}. Domain averaged surface values, liquid water paths, cloud fraction etc. Output and sampling done every \code{ssam\_intvl}. Needs to be post processed for MPI runs.
 \end{itemize}
\end{frame}
\begin{frame}[allowframebreaks]{Statistics and output}
\mylineno=0\begin{longtable}{p{0.15\linewidth}p{0.15\linewidth}p{0.6\linewidth}}
\alert{Variable} &\alert{Default} & \tblnewline 
\endhead
outflg    &  true   &  output flag (true/false) \tblnewline
lsync & false &  Synchronize the crosssection output (true/false)\tblnewline
frqhis    &  9000 s &  history write interval \tblnewline
frqanl    &  3600 s &  analysis write interval  \tblnewline
slcflg    &  false  &  write slice output (true/false) \tblnewline 
istpfl    &  1 &  print interval for timestep info \tblnewline
ssam\_intvl&    30 s  &  statistics sampling interval\tblnewline
savg\_intvl&  1800 s    &  statistics averaging interval \tblnewline
lcross & false &  Crosssection output (true/false)\tblnewline
frqcross    &  3600 s &  crosssection write interval  \tblnewline
lxy & false &  Crosssection output in xy plane (true/false)\tblnewline
zcross & 0 &  Crosssection location of xy plane (true/false)\tblnewline
lxz & false &  Crosssection output in xz plane (true/false)\tblnewline
ycross & 0 &  Crosssection location of xy plane (true/false)\tblnewline
lyz & false &  Crosssection output in yz plane (true/false)\tblnewline
xcross & 0 &  Crosssection location of xy plane (true/false)\tblnewline
lwaterbudget & false &  Crosssection of (costly) waterbudget (true/false)\tblnewline
\end{longtable}
\end{frame}

\begin{frame}{Timeseries}
\begin{itemize}
 \item  Postprocessing to make 1 file out of all the files per processor
 \item Build tool in \code{uclales/misc/synthesis}: 
 \item \code{ifort reducets.f90 -o reducets \textasciigrave /path/to/netcdflib/bin/nc-config --fflags --flibs \textasciigrave }
 \item \alert{NOTE: The quotation marks are accent graves (Under the tilde at a US International keyboard}
  \item Use it to gather your timeseries statistics with:
\code{reducets name nx ny}
\begin{itemize}
 \item \code{name} is the \emph{stem} of the filename (so everything before \code{.ts.00....}
 \item  \code{nx} is the number of processes in the x-direction
 \item \code{ny} is the number of processes in the y-direction
\end{itemize}
\end{itemize}
\end{frame}

\begin{frame}{Profiles}
\begin{itemize}
 \item  Postprocessing to make 1 file out of all the files per processor
 \item Build tool in \code{uclales/misc/synthesis}: 
 \item \code{ifort reduceps.f90 -o reduceps \textasciigrave /path/to/netcdflib/bin/nc-config --fflags --flibs \textasciigrave }
 \item \alert{NOTE: The quotation marks are accent graves (Under the tilde at a US International keyboard}
  \item Use it to gather your profile statistics with: \code{reduceps name nx ny}
\begin{itemize}
 \item \code{name} is the \emph{stem} of the filename (so everything before \code{.ps.00....}
 \item  \code{nx} is the number of processes in the x-direction
 \item \code{ny} is the number of processes in the y-direction
\end{itemize}
\end{itemize}
\end{frame}
\begin{frame}{Adding to Profiles and Timeseries}
 \begin{itemize}
  \item Both profiles and timeseries are written from \code{ncio.f90} and \code{stat.f90}
  \item They are known to change over time. 
 \end{itemize}

\end{frame}

\begin{frame}{Plot}
 \begin{itemize}
  \item You're completely free to do what you want :)
  \item Depending on who you are and what you want for a plot, you could use NCL, Matlab, Python, Ferret, NCView,...
  \item We'd like to build up a tools database, so feel even more free to submit scripts over git
  \item As a starter, copy the 2 \code{plotfld.*} scripts from \code{uclales/misc/analysis/}
  \item Explore \code{plotfld.csh}, and put in the right variable names and time frame.
  \item Run it!
  \item Output sits in two pdf files \code{t1.pdf} and \code{p1.pdf}
 \end{itemize}
\end{frame}

\begin{frame}{Crosssections}
\begin{itemize}
 \item Postprocessing to make 1 file out of all the files per processor:
\item \code{cdo gather name.out.cross*nc name.out.cross.nc}
\item Watch the file quickly with (for instance) \code{ncview}
\end{itemize}

 

\end{frame}
